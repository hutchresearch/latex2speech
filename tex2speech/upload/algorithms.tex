
\section{Reconstruction of \texorpdfstring{\bigmet}{MET}}
\label{sec:algorithms}

In general, \vecmet is the negative of the vector
sum of the transverse momenta of all
final-state particles reconstructed in the detector.
CMS has developed three distinct algorithms to reconstruct \vecmet\ :
(a) \pfmet, which is calculated using a complete
particle-flow technique~\cite{PFT-09-001};
(b) \calomet, which is based on calorimeter energies and the calorimeter
tower geometry~\cite{METPas07}; and
(c) \tcmet, which corrects \calomet\ by including tracks reconstructed
in the inner tracker after  correcting for the tracks' expected energy
depositions in the calorimeter~\cite{JME-09-010}.






\pfmet is calculated from the reconstructed PF particles.
\pfsumet is the associated scalar sum of the transverse energies of the PF particles.



\calomet is calculated using the energies contained in calorimeter towers and their
direction, relative to the centre of the detector,
to define
pseudo-particles.  The sum excludes energy deposits
below noise thresholds.
Since a
muon deposits only a few GeV on average  in the calorimeter,
independent of its momentum, the muon \pt
is included in the \calomet calculation
while the small calorimetric energy deposit
associated to the muon track is excluded.
\calosumet is the associated scalar sum of the transverse energies of the
calorimeter towers and muons.


\tcmet
is based on \calomet, but also
includes the $p_{\rm T}$s of tracks that have been
reconstructed in the inner tracker, while removing the expected
calorimetric energy deposit  of each track.
The predicted energy deposition for charged pions is used for all
tracks not identified as electrons or muons.
The calorimetric energy deposit is
estimated from simulations of single pions, in intervals of $p_{\rm T}$ and
$\eta$, and an extrapolation of the track in the CMS
magnetic field is used to determine its expected position.
No correction is applied for very high $p_{\rm T}$ tracks
($p_{\rm T}>100~\GeV$), whose energy is already well measured  by the
calorimeters. For low-$p_{\rm T}$ tracks ($p_{\rm T}<2~\GeV$), the measured
momentum is taken into account
assuming
no response from the calorimeter.






The magnitude of the \vecmet
can be underestimated for a
variety of reasons,
including
the nonlinearity of the response of the calorimeter
for neutral and charged hadrons due to its noncompensating nature,
neutrinos from semileptonic decays of particles,
minimum energy thresholds in the calorimeters, \pt thresholds and inefficiencies
in the tracker, and, for \calomet, charged particles that are bent
by the strong magnetic field of the CMS solenoid and whose
calorimetric energies are therefore in a calorimeter cell whose associated
angle
is very different from the angle of the track at the vertex.
The displacement of charged particles with small \pt due to
the magnetic field and the
calorimeter nonlinearity are the largest of these biases,
and thus \calomet\ is affected most.
A two-step correction has been devised in order to remove the bias in
the \vecmet\ scale.  The correction procedure relies on the fact that
\vecmet can be factorized into contributions from jets,
isolated high \pt photons, isolated high \pt electrons, muons, and unclustered energies.
The contribution due to unclustered energies is the difference between the \vecmet
and the negative of the vector sum of the $p_{\rm T}$s of the other objects.
Isolated photons, electrons, and muons are assumed to require no scale corrections.


Jets can be corrected to the particle
level using the jet energy correction~\cite{JME-10-003}.
The ``type-I corrections'' for
$\vecmet$ use these
jet energy scale corrections for
all jets that have less than $0.9$ of their energy in the ECAL
and corrected \pt$>20$~GeV for \calomet,
and for a user-defined selection of jets with \pt$>10$~GeV for \pfmet.
These corrections can be up to a factor of two for \calomet\
but are less than 1.4 for \pfmet \cite{JME-10-014} .
In order to correct the remaining soft jets below this threshold,
and energy deposits not clustered in any jet, a
second correction can be applied to the unclustered energy, which is
referred to as the ``type-II correction''.  This correction is obtained from
$\cPZ\to\Pe\Pe$ events, as discussed in the Appendix.

In this paper, distributions involving \calomet\ include both type-I and type-II corrections,
those involving \pfmet\ include type-I corrections, and those involving
\tcmet\ are uncorrected, as these were the corrections that were available at the time the analyses
presented in this paper were performed and are the versions used most typically
in 2010 physics analyses.
As discussed in the Appendix, type-II corrections have been developed for \pfmet
and can be used in future analyses.
The optimization of both corrections is also discussed in the Appendix.


