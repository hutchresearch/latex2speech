\section{Business Requirements}
\subsection{Background}
% What is the business context for the project?
% What does the reader need to know about the customer's industry?
For every working researcher, staying up to date with all the newest developments in their field of study is crucial. Since the goal of research is to build upon previous discoveries to make new ones, the first step is to know what has already been discovered. This is for the most part accomplished by reading papers published by other researchers in the same field and must be done continuously. Therefore, increasing the possible times to read papers, and in general increasing their accessibility is important enough to warrant the development of new tools to aid in this.

\subsection{Business Opportunity}
% What is the customer's need?
% How does this need fit in the industrial context?
Our customer has found that time spent driving could be used as a time to read additional papers if they could be read aloud. As the popularity of audiobooks, podcasts, and other spoken media shows, this is not an uncommon need at all and extends beyond driving to simply any time where our minds are idle but our hands are occupied. In the context of research, extending the ability to read papers to these occasions would be a boon for any working researcher.

\subsection{Business Objectives and Success Criteria}
% Why does the customer need this product?
% How will the customer know that the product is a success?
%   You must be very specific here.  What experiment can we do to verify success?
\par
Unfortunately, unlike more common forms of text media, scientific papers have no real good option for being read aloud. Traditional text-to-speech (TTS) systems have made most traditional texts available as audio, but the complexities found in most technical writing translate poorly (or not at all in the case of mathematical equations) to an audio equivalent. Other products address this problem, but they have a limited feature set that makes reading an entire paper from top-to-bottom difficult if not impossible.\\
\par
\noindent Our solution is \TeX 2Speech, a program that will parse the most common language for technical writing, \LaTeX\, into a form that can be read appropriately by existing TTS systems. To succeed, \TeX 2Speech will need to parse any arbitrary \LaTeX\ file and give the user a fully rendered audio output of the document as a result. It will have to remove marks not relevant for an understanding of the text, while also being able to indicate proper phrasing of the text, especially mathematical expressions that existing TTS systems have trouble with. To increase accessibility, the project will be available as a service over the internet, and users will interact with \TeX 2Speech by uploading their \LaTeX\ files and downloading the resulting audio.

\subsection{Customer and Market Needs}
% Connect the objectives and success criteria back to the customer's business.
% That is, why does the industry require the customer to have this solution? 
\TeX 2Speech would provide the solution to our customer’s desire for reading papers while otherwise indisposed. As was noted before, any working researcher needs to be continuously up to date on the latest relevant papers, and our solution will provide an avenue to do so more efficiently. For those with long commutes, unavoidable delays, or who simply enjoy audio media in the many forms it already exists in, \TeX 2Speech will extend that option to technical writing in its many forms.

\subsection{Business Risks}
% This is a hazard assessment.  What could go wrong?  How bad would it be if it
% did?  How likely is it?  What steps can we take to protect against the hazard? 
If users try to open up a gigantic markup \LaTeX\ file, depending on the implementation of \TeX 2Speech this could crash the program. While most \LaTeX\ documents in our use case are reasonably sized, open-source textbooks make this a real if unlikely concern and the consequences could be severe (crashing, non-functional program). \\

\noindent \LaTeX\ is a very flexible language, so users could reasonably use a package or command we didn’t take into account. This would be a high probability risk, with at least a medium severity since every use of the command would have a chance to have our TTS system say something with no equivalent English meaning.\\

\noindent Regardless of what we are parsing, there is always the risk of mis-parsing or mis-pronouncing any number of symbols. While the likelyhood of this happening would depend on both the file being parsed and our implementation, the severity would either be low to medium depending on the importance of the given word or symbol being pronounced.\\

\noindent Finally, the third-party software we’re using might be shut down or not online for some time. This would be low probability since the odds of a third party shutting down usually only happens in maintenance or a problem on their ends. This would also be low severity for us since we can just go to another third party software. 

