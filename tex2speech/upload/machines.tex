\section{Accelerator Parameters}

There are several $\epem$ linear collider design efforts currently 
underway, differing most significantly in the choice of RF 
acceleration.
The TESLA design from DESY uses superconducting RF cavities with 
resonant frequency of 1.3~GHz.
The NLC/JLC-X design from SLAC and KEK uses normal conducting X-band
cavities with resonant frequency of 11.4~GHz.
The JLC-C design from KEK uses normal conducting C-band cavities with 
resonant frequency of 5.7~GHz.
These three designs all use klystrons as the RF power source.
The CLIC R\&D program at CERN uses 30~GHz normal conducting 
cavities, coupled to a drive beam linac for power.
The choice of the RF acceleration method causes differences in the 
bunch structure parameters of the various designs.
We will not go into detail on the designs here but just touch briefly 
on the time structure, energy, and luminosity.
Table~\ref{table-lc_params} summarizes these parameters.
%
\begin{table}[t]
\centering
\caption{Baseline parameters for four linear collider designs.}
\label{table-lc_params}
\vskip 6pt
\begin{tabular}{lcccc}
\hline \hline
  & TESLA & NLC/JLC-X & JLC-C & CLIC \\
\hline 
$\sqrt{s}$ (TeV) & 0.5--0.8 & 0.5--1.0 & 0.5 & 0.5--3 \\ 
${\cal L}$ (10$^{34}$ cm$^{-2}$ s$^{-1}$)
	& 3.4--5.8 & 2.2--3.4 & 0.43 & 1--10 \\
RF cavities & superconducting & normal & normal & normal \\
RF power source & klystrons & klystrons & klystrons & drive beam \\
bunches/train & 2820--4886 & 190 & 72 & 150 \\ 
bunch separation (ns) & 337 & 1.4 & 2.8 & 0.7 \\
repetition rate (Hz) & 5  & 120  & 50 & 200  \\ 
\hline \hline
\end{tabular}
\end{table} 
More detailed descriptions and references can be found in
Refs.~\cite{Kuhlman:1996rc,Toge:2000mm,Brinkmann:1999rz,Assmann:2000hg}.

All designs plan to have polarized $e^-$ beams, with $P_{e^-}=80$\%.
There is also some work on polarized $e^+$ sources, achieving perhaps 
$P_{e^+}=40$\% at full luminosity and 60\% with reduced
luminosity~\cite{Walker}.
With both beams polarized the effective polarization for $e^+e^-$ 
annihilation processes is 
$P_{\rm eff}=(P_{e^-}+P_{e^+})/(1 + P_{e^-}P_{e^+})$.

\subsection{TESLA}

The TESLA design uses superconducting cavities, operated at 2 K,
with a resonant frequency of 1.3~GHz.  The baseline design calls for
$\sqrt{s}=500$ GeV, with an upgrade path to 800~GeV.  For the baseline
design, the beam structure will be long bunch trains of 2820 bunches
separated by 337~ns, for a total bunch train length of 950 $\mu$s
(285 km) and a 5 Hz repetition rate.  The nominal luminosity at 500~GeV
is $3.4\times 10^{34}$~cm$^{-2}$~s$^{-1}$. The higher energy
requires higher gradient from the superconducting cavities~\cite{Tesla:II}.

\subsection{NLC and JLC}

The unified NLC/JLC-X design uses normal conducting cavities with
a resonant frequency of 11.4~GHz.
The baseline design calls for $\sqrt{s}=500$~GeV,
with an upgrade path to 1~TeV~\cite{NLC:Snowmass}.
For the present baseline design, the beam
structure will be bunch trains of 190 bunches separated by 1.4~ns,
for a total bunch train length of
0.27~$\mu$s (81~m) and a repetition rate of 120~Hz.  The nominal
luminosity is $2.2\times 10^{34}$~cm$^{-2}$~s$^{-1}$.  The baseline
option only fills half the linac tunnel with RF cavities, enabling
a straightforward upgrade to 1~TeV.

The JLC collaboration is also pursuing an accelerator based on C-band 
(5.7~GHz) as a backup for the X-band design.
Most of the components of the C-band main linac satisfy the
specifications of the 500~GeV JLC.
The C-band design is considered by some to be a serious option, if one
wants to build a normal-conducting machine as early as possible.

\subsection{CLIC}

The CLIC design uses normal conducting cavities with a resonant
frequency of 30 GHz.  Compared to the other designs, it is earlier
in the R\&D phase.  The novel approach of CLIC is that the
RF power is delivered to the accelerating cavities by a drive beam
with coupled cavities.  The design is being optimized for 3~TeV,
but the concept is being developed for 0.5--5~TeV.  For the baseline
design, the beam structure will be bunch trains of 150 bunches
separated by 0.7~ns, for a total bunch train length of 0.1~$\mu$s
(30 m) and a repetition rate of 200~Hz.  The nominal luminosity is
$10^{34}$--$10^{35}$~cm$^{-2}$~s$^{-1}$.


\subsection{Detector Backgrounds}

\par  Studies have been done of backgrounds in the detectors from
extra $\epem$ pair production.  Since these are generally low momentum
electrons, they could spiral in the central magnetic fields, causing
larger occupancies in the tracking detectors.  
Because the TESLA and
NLC/JLC designs have very different time structures, we have 
investigated the density of
extra hits per bunch crossing (for TESLA) or bunch train (for NLC/JLC).

\par  With the 337~ns bunch spacing in the TESLA design, individual
bunch crossings can be resolved with the detector readout electronics. 
Therefore, it is appropriate to consider hits per bunch crossing 
as a measure of background hits.
With the bunch spacing of 2.8~ns in the (earlier) NLC/JLC design, 
the detector readout electronics will most likely integrate 
95 bunches, and the relevant unit for background measure is 
number of hits per bunch train.
Simulations done in the ECFA-DESY Study~\cite{ref-Eur_density},
for $\sqrt{s}=500$ GeV and a central magnetic field 3~T,
give 0.2--0.5~hits/mm$^2$/bunch at a radius of 1.2~cm; see
Fig.~\ref{1:fig:MB}.
%
\begin{figure}[bp]
\centering
	\includegraphics[width=0.7\textwidth]{pairs_z.eps}
	\caption[1:fig:MB]{The background hit densities at TESLA in each 
	vertex layer as a function of a distance along the beamlines for 
	3~T magnetic field.  From Ref.~\cite{ref-Eur_density}.}
	\label{1:fig:MB}
\end{figure}
%
In contrast, simulations done for the
NLC/JLC VXD~\cite{ref-VXD_density} result in 3 hits/mm$^2$/train
for a 3~T central magnetic field (or 0.03 hits/mm$^2$/bunch).
Figure~\ref{2:fig:TWM} shows the results of these simulations, 
where number of background hits per bunch is plotted 
as a function of the magnetic field at various radii. 
(The 280 hits in Layer 1 for the magnetic field of 3 T, 
shown in Figure~\ref{2:fig:TWM}, correspond to a hit density of 
 ~3 hits/mm$^2$/train.) 
%
\begin{figure}
\centering
	\includegraphics[width=0.5\textwidth]{vxd_hits.eps}
	\caption[2:fig:TWM]{The raw number of hits at NLC in each layer 
	as a function of magnetic field for the small detector.  (The
	masking layout was not modified for the low magnetic field, so 
	hits from pairs which impacted the outer face of the M1 mask were
	deleted.)  From Ref.~\cite{ref-VXD_density}.}
	\label{2:fig:TWM}
\end{figure}

Both groups are working on designs of vertex
detectors with a first active layer placed at a radial distance of
1.2~cm away from the colliding beams.
For both bunch time structures, it is felt
that it is possible to operate such detectors with hit densities
seen in the simulations.

In addition to beam-induced background sources discussed in the 
previous studies, overlapping hadronic $\gamma\gamma$ interactions 
provide a source of background.
Preliminary studies of their effect on reconstruction of 
physics processes of interest, {\it e.g.}, for Higgs decays, 
have been done in the ECFA-DESY Study~\cite{ref-MB_ggbkg}. 
The authors concluded that a combination
of kinematic and vertex topology selections can reduce the effects of 
$\gamma\gamma$ interactions, with moderate losses in reconstruction 
efficiency.  The background events resulted in a very small 
additional number of charged hits in the inner layer of the vertex 
detector in the amount of $3.4 \times 10^{-5}$~hits/mm$^2$/bunch.

Optimization work on the mask designs is still in progress, to reduce
the beam backgrounds even further.
In addition, there are other ideas leading to background reduction,
which remain to be evaluated.
One possibility, currently under study, is to increase the strength of
the magnetic field in the detector.
Another option, relevant for the NLC/JLC beam structure, is to
reduce the number of bunches recorded by the electronics during the
collisions.
Such reduction can be achieved by applying pipelined front-end readout 
with a length of the integration window shorter than the 266~ns 
duration of the NLC/JLC bunch train.
However, the presently achieved reduction factors are very good for 
both machine designs and result in acceptable levels of background.

