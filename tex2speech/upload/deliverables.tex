\section{Deliverables}
% The purpose of this section is to formally list the major components that you
% will produce as part of this project.  

\subsection{Software/Hardware}
% List the software components and hardware, if any, that you will deliver as
% part of this project.  
\underline{\LaTeX\ to SSML Converter}: This is the largest and most fundamental piece of software we will be developing. It will contain all the rules for how different \LaTeX\ commands/syntax matches up with specifics of prosody and structure so proper pronunciation can be achieved. This piece of software will likely be further broken into several sub-modules to help aid in smooth development and managing the complexity of the task, but further research will be required to know where those distinctions will be drawn.\\
\newline
\underline{\TeX 2Speech Server}: This will be the request handling portion of the \TeX 2Speech system. It will manage interactions between users (their inputs and any files uploaded), the \LaTeX\ to SSML Converter and the TTS service upstream. While substantially simpler than the \LaTeX\ to SSML converter, its creation will still be involved and will determine how accessible our software is to use.

\subsection{Documentation}
% What user documentation will you deliver.
This project will include documentation regarding what features of \LaTeX\ \TeX 2Speech can support, including notable external packages and any limitations. It will specify the behavior upon encountering an error or unknown token, as well as acknowledgments of possible undefined edge cases. Along with this will be more basic documentation about how to navigate the program via its UI to upload \LaTeX\ files and get an audio output.

\subsection{Key Presentations}
% You will present your project at the end of 492 and 493.
\underline{End of CS 492}: We aim to present a program that converts \LaTeX\ to SSML. At this stage it will be invoked from a terminal. We will demonstrate the conversion of key \TeX\ functions to SSML using a predefined dictionary and analysis of document structure. We will then run this SSML through Amazon Polly to produce synthesized speech. \\
\newline
\underline{End of CS 493}: We will present document conversion using a more complete \TeX\ to SSML dictionary, and more sophisticated format analysis. Additionally, we will present a web application that allows users to easily convert LaTeX documents into WAV files.